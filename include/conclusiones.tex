\begin{conclusion}

% Breve resumen del trabajo realizado
% Recuento de objetivos alcanzados y no alcanzados
% Análisis crítico de por qué los resultados fueron los reportados
% Reflexión acerca de la relevancia / impacto del trabajo realizado
% Lecciones aprendidas
% Posibles trabajos futuros que podrían hacerse a partir de la memoria para mejorar aún más la solución

A lo largo de este trabajo se realizó la implementación de un sistema de aprendizaje de reglas de asociación, o \textit{Association Rule Learning (ARL)}, para grandes conjuntos de transacciones. El sistema permite al usuario generar reglas que cumplan con medidas mínimas de relevancia estadística, tales como soporte y confianza, y posteriormente ser rankeadas según estas mismas medidas. Junto con esto, el usuario es capaz de requerir al sistema que despliegue solamente aquellas reglas en las que esté presente un cierto ítem en su antecedente o consecuente; generando, de esta manera, más valor a los resultados en vista de su relevancia para el usuario.

En particular, se enfocó su uso a datos provenientes de mediciones espectroscópicas astronómicas; con el fin de encontrar asociaciones lógicas entre líneas espectrales. Para ello se procedió a realizar pruebas de concepto sobre una base de datos de espectros ópticos del \textit{Sloan Digital Sky Survey (SDSS)}, previo un pre-procesamiento y análisis de los datos. Los resultados obtenidos mostraron que el sistema cumple con sus objetivos principales; que son el generar asociaciones lógicas entre conjuntos de líneas espectrales, y desplegarlas al usuario según criterios de relevancia.

La relevancia y el impacto del presente trabajo se aprecia mejor en el marco de proyectos como el del \textit{Atacama Large Millimeter Array (ALMA)}, en el cual, dentro de los próximos años, se comenzará a generar grandes cantidades de datos de espectroscopía astronómica, y que estos se acumulen con el tiempo. El hecho de que muchos de estos datos se obtengan como consecuencia, y no como objetivo principal de muchas de las observaciones por parte de los astrónomos, es un indicador de la importancia de tener herramientas computacionales que permitan auxiliar al proceso de investigación y que disminuyan los requerimientos de horas-hombre necesarios para realizar descubrimientos de interés.

A lo larto de este trabajo se logró aprender detalles muy importantes del proceso de implementar una herramienta que utiliza algoritmos y métodos generales a una solución específica, en un dominio del conocimiento muy teórico y de lenguaje muy técnico, como es el de la astronomía. Se pudo asimilar lo que implica hacer un proceso de investigación previo a la fase misma de implementación de un sistema, con el fin de que sus prestaciones se encuentren alineadas con sus requerimientos. Y esto se vuelve aun más crucial en aplicaciones científicas interdisciplinarias. La interacción con expertos de diversas ramas del conocimiento y la investigación fue, sin lugar a dudas, uno de los puntos más importantes en el proceso de aprendizaje llevado a cabo en el desarrollo de este trabajo.

Quera para el desarrollo a futuro el optimizar el flujo de trabajo de la herramienta, mediante hacer más compacta las interfaces entre módulos y hacer más general la aplicación del sistema de pre-procesamiento de datos; con el fin de que se vuelva parte íntegra del sistema, replicable y adaptable por el usuario a datos de características diversas. 

Otro importante objetivo que queda para futuro es la implementación de una interfaz gráfica de usuario, que facilite la visualización, manejo de resultados y el evitar labores repetitivas por parte del usuario en su flujo de trabajo. Una alternativa a este punto sería hacer que el sistema sea parte de alguna herramienta ya existente de visualización y operación de datos astronómicos.

Relacionado con esto está otro importante objetivo a futuro, que es el realizar la implementación de la herramienta en ambientes de computación de alto rendimiento, y el hacer que se conforme a estándares de observatorios virtuales. Medidas como estas expandirán de forma considerable las posibles aplicaciones futuras y el impacto de la solución desarrollada a lo largo de este trabajo.

\end{conclusion}