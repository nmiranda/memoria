%\section{Introducción}
\begin{intro}

% Resume todo pero en más detalle
% - Contexto
% - Problema
% - Relevancia / Motivación para encontrar una solución
% - Alternativas analizadas, alternativa escogida
% - Descripción general de la solución
% - Resultados de la solución implementada para resolver el problema



%\bookepigraph{10cm}{we may in time ascertain the mean temperature of heavenly bodies, but I regard this order of facts as for ever excluded from our recognition. We can never learn their internal constitution}{Auguste Comte}{Astronomy, Ch. I: General View, 1835}

\vspace{1em}
\hfill{}
\begin{minipage}{9cm}{
\begin{spacing}{0.9}
\small
\noindent
\textit{[...] we may in time ascertain the mean temperature of heavenly bodies, but I regard this order of facts as for ever excluded from our recognition. We can never learn their internal constitution [...]}
\end{spacing}
\vspace{1em}
\hfill{}{Auguste Comte, \textit{Astronomy, Ch. I: General View}, 1835}
}
\vspace{2em}
\end{minipage}

En los últimos tiempos, y en gran parte debido al explosivo desarrollo tecnológico, han surgido numerosos campos en los cuales se ha requerido el uso de procesamiento masivo de datos e inteligencia computacional con el fin de automatizar y auxiliar el proceso de generación de nuevo conocimiento. La astronomía es, sin lugar a dudas, uno de ellos. Esto se debe, en parte, al explosivo desarrollo de nuevas tecnologías que ponen al alcance de la comunidad científica una cantidad nunca antes vista de datos; los cuales tienen el potencial de contener invaluable información sobre el universo, su composición, estructura, origen y destino.

Un claro ejemplo de esto lo constituye el \textit{Atacama Large Millimiter/sub-millimiter Array (ALMA)}, un interferómetro radio-astronómico que consiste en un arreglo de 66 antenas que observan el espacio en las bandas milimétricas y submilimétricas del espectro electromagnético. Ubicado en el desierto de Atacama, en el norte de Chile, es parte de uno de los proyectos científicos más importantes del último tiempo a nivel nacional; en el cual se ha hecho uso de tecnologías de punta por parte de investigadores, ingenieros y técnicos expertos en computación de alto rendimiento, redes de fibra óptica, Machine Learning, minería de datos, entre otros. 

La tecnología involucrada en el proyecto \textit{ALMA} ha permitido, entre otras cosas, obtener datos de alta resolución provenientes de distintas fuentes u objetos del espacio observable desde la tierra, cuya posición en el cielo es cuantificada mediante coordenadas celestes. La radiación electromagnética emitida por estos objetos, en bandas de frecuencia de radio, son captadas por el arreglo de antenas y posteriormente procesadas por equipos de alta capacidad con el fin de obtener los espectros electromagnéticos correspondientes. Estos, a su vez pueden ser analizados directamente o utilizarse para generar imágenes de alta calidad de regiones del espacio en diversos rangos de frecuencia.

Si bien el caso de \textit{ALMA} es notable por las características particulares de las observaciones en frecuencias de radio y por las oportunidades tecnológicas que involucra, el analizar y extraer información a partir de radiación electromagnética, en diversos rangos de su espectro, es parte primordial de la labor astronómica en todos sus campos.

Parte principal de la importancia de estos espectros de radiación electromagnética es que dan información valiosa sobre la composición química de los objetos de los que esta proviene; ya sean estrellas, galaxias u otros muchos tipos de estructuras celestiales. Esto se debe a que los átomos que componen estos objetos emiten o absorben una mayor cantidad de energía en frecuencias muy específicas, tal y como lo predice la teoría cuántica subyacente. Por lo tanto, un espectro en particular tendrá puntos más altos (picos) en ciertas frecuencias dependiendo de los elementos químicos de los que está compuesto el objeto del que proviene.

La detección de lineas espectrales es un problema de interés en sí, y que puede llegar a ser muy complejo dependiendo de en qué bandas de frecuencia se esté trabajando. Sin embargo, se puede seguir obteniendo información valiosa de los objetos observados a partir de estas líneas ya detectadas. Esto incluye potencialmente respuestas a preguntas como: ¿de qué forma se relacionan ciertos tipos de líneas entre sí? ¿Existe una mayor correlación de presencia de líneas de ciertos isótopos o moléculas en particular? ¿Hay una mayor presencia de líneas de ciertas especies en algunos objetos que en otros? ¿Qué nos dice esto de la composición de los objetos y de su química subyacente?


\end{intro}