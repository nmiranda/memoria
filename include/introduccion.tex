%\section{Introducción}
\begin{intro}

% Resume todo pero en más detalle
% - Contexto
% - Problema
% - Relevancia / Motivación para encontrar una solución
% - Alternativas analizadas, alternativa escogida
% - Descripción general de la solución
% - Resultados de la solución implementada para resolver el problema

\vspace{1em}
\hfill{}
\begin{minipage}{9cm}{
\begin{spacing}{0.9}
\small
\noindent
\textit{[...] we may in time ascertain the mean temperature of heavenly bodies, but I regard this order of facts as for ever excluded from our recognition. We can never learn their internal constitution [...]}
\end{spacing}
\vspace{1em}
\hfill{}{Auguste Comte, \textit{Astronomy, Ch. I: General View}, 1835}
}
\vspace{2em}
\end{minipage}

\section*{Contexto y motivación}
\addcontentsline{toc}{section}{Contexto y motivación}

En los últimos tiempos, y en gran parte debido al explosivo desarrollo tecnológico, han surgido numerosos campos en los cuales se ha requerido el uso de procesamiento masivo de datos e inteligencia computacional con el fin de automatizar y auxiliar el proceso de generación de nuevo conocimiento. La astronomía es, sin lugar a dudas, uno de ellos. Esto se debe, en parte, al explosivo desarrollo de nuevas tecnologías que ponen al alcance de la comunidad científica una cantidad nunca antes vista de datos; los cuales contienen abundante información sobre el universo, su composición, estructura, origen y destino.

Un claro ejemplo de esto lo constituye el \textit{Atacama Large Millimiter/sub-millimiter Array (ALMA)} \cite{wootten2009atacama}, un interferómetro radio-astronómico que consiste en un arreglo de 66 antenas que observan el espacio en las bandas milimétricas y submilimétricas del espectro electromagnético. Ubicado en el desierto de Atacama, en el norte de Chile, es parte de uno de los proyectos científicos más importantes del último tiempo; en el cual se ha hecho uso de tecnologías de punta por parte de investigadores, ingenieros y técnicos expertos en diversas áreas del conocimiento, tales como la astronomía, la computación científica y de alto rendimiento, la electrónica, entre otros.

La tecnología involucrada en el proyecto \textit{ALMA} ha permitido, entre otras cosas, obtener datos de alta resolución provenientes de distintas fuentes u objetos del espacio observable desde la tierra. La radiación electromagnética emitida por estos objetos, en bandas de frecuencia de radio, son captadas por el arreglo de antenas y posteriormente procesadas por equipos de alta capacidad con el fin de obtener los espectros electromagnéticos correspondientes. Estos, a su vez pueden ser analizados directamente o utilizarse para generar imágenes de alta calidad.

Parte principal de la importancia de estos espectros de radiación electromagnética es que dan información valiosa sobre la composición química de los objetos de los que esta proviene. Esto se debe a que los átomos que componen estos objetos emiten o absorben una mayor cantidad de energía en frecuencias muy específicas. Por lo tanto, un espectro en particular tendrá rangos estrechos de mayor o menor intensidad en ciertas frecuencias dependiendo de los elementos químicos de los que está compuesto el objeto del que proviene.

La detección de lineas espectrales es un problema de interés en sí, y que puede llegar a ser muy complejo dependiendo de en qué bandas de frecuencia se esté trabajando. Sin embargo, se puede seguir obteniendo información valiosa de los objetos observados a partir de estas líneas ya detectadas. Esto incluye potencialmente respuestas a preguntas como: ¿de qué forma se relacionan ciertos tipos de líneas entre sí? ¿Existe una mayor correlación de presencia de líneas de ciertos isótopos o moléculas en particular? ¿Hay una mayor presencia de líneas de ciertas especies en algunos objetos que en otros? ¿Qué nos dice esto de la composición de los objetos y de su química subyacente?

Existe, en el dominio de la minería de datos, el concepto de \textit{reglas de asociación}; las cuales corresponden a asociaciones lógicas entre conjuntos de elementos o ítems. Si bien estos inicialmente fueron concebidos con el fin de resolver problemas pertenecientes al ámbito del comercio y las ventas, hoy en día son aplicados en los más diversos contextos. Es, por lo tanto, una de las finalidades principales de este trabajo, el mostrar que la espectroscopía astronómica no es la excepción, y que es posible extraer reglas de asociación entre líneas espectrales obtenidas a partir de observaciones de objetos del espacio.

Se espera que el generar de manera automática reglas de asociación entre líneas moleculares facilite, a futuro, el análisis de la naturaleza de los objetos observados, y las caracterísicas de su composición química; al averiguar cómo se relacionan entre sí los elementos, átomos, moléculas e isótopos presentes en las sustancias que los componen. Esto, sobre todo, en vista de que hoy en día el volumen de datos generados a partir de observaciones astronómicas no deja de aumentar.

\section*{Objetivos}
\addcontentsline{toc}{section}{Objetivos}

\subsection*{Objetivo General}
\addcontentsline{toc}{subsection}{Objetivo General}

\begin{itemize}
	\item Implementar un sistema de aprendizaje de reglas de asociación, o \textit{Association Rule Learning (ARL)}, que permita obtener relaciones lógicas entre líneas espectrales presentes dentro de un conjunto de datos de espectroscopía astronómica.
\end{itemize}

\subsection*{Objetivos Específicos}
\addcontentsline{toc}{subsection}{Objetivos Específicos}

\begin{itemize}
  \item Implementar un sistema de ARL genérico que permita aplicarse a datos provenientes de diversos orígenes.
  \item Obtener reglas de asociación entre líneas espectrales obtenidas a partir de datos reales.
  \item Visualizar las reglas de asociación, presentes en el conjunto de datos, que sean de mayor interés según medidas estadísticas. 
  \item Filtrar las reglas de asociación encontradas en un conjunto de datos de espectroscopía astronómica según las líneas que las componen.
\end{itemize}

\section*{Descripción de la solución}
\addcontentsline{toc}{section}{Descripción de la solución}

Si bien existen diversas técnicas de clasificación y caracterización de puntos en un espacio multidimensional (en nuestro caso objetos descritos por parámetros), para resolver las preguntas anteriores se requiere más bien de una herramienta que permita encontrar relaciones explícitas entre los parámetros en sí, y que permita asignar medidas de relevancia estadística a estas relaciones.

Para ello se planteó el uso de \textit{Association Rule Learning (ARL)}, o Aprendizaje de Reglas de Asociación, como una herramienta que puede dar respuesta directa a algunas de las interrogantes mencionadas anteriormente, y ayudar a obtener información clave para el proceso de utilizar otras técnicas en el largo plazo.

El Aprendizaje de Reglas de Asociación como técnica se ubica dentro del área de la minería de base de datos, y su concepción original fue el ser aplicada a sistemas de puntos de venta con el fin de encontrar las relaciones más comunes entre artículos comprados por los clientes. Sin embargo, con el tiempo se ha convertido en una de las herramientas más utilizadas de su área, en una diversa gama de contextos.

En el presente trabajo se llevó a cabo el uso de esta técnica con el fin de encontrar relaciones comunes entre líneas espectrales a través de distintos espectros de frecuencia. Ahora bien, la naturaleza innata de estos es más bien contínua y las líneas en sí mismas poseen diversos parámetros que las caracterizan. Por lo tanto, este caso dista mucho de la binaridad del problema original para el cual se pensó ARL. Sin embargo, como se muestra a lo largo de este informe, si se asume que se realizó con anterioridad un buen trabajo de detección de líneas y se efectúa un pre-procesamiento adecuado de los datos, el algoritmo de ARL arroja resultados que están en concordancia con la química subyecente.

En particular, se utilizó una implementación de dos de los algoritmos más utilizados de ARL: \textit{Apriori} y \textit{FP-Growth}. Luego, se obtuvo una base de datos de líneas espectrales ya detectadas (pero no necesariamente asociadas a alguna especie [átomo, isótopo, etc.]) correspondientes a observaciones del \textit{Sloan Digital Sky Survey (SDSS)}, un sondeo espectroscópico del espacio realizado con un telescopio óptico. Sobre este conjunto de datos se procedió a realizar un pre-procesamiento que, entre otros, consta de filtrar las líneas según su brillo o razón señal a ruido. Luego, se efectuaron particiones según las características de los objetos de procedencia (como tipo de objeto o estructura estelar, cercanía, etc.). Finalmente, sobre estas se procedió a aplicar los algoritmos de ARL.

Los resultados obtenidos fueron efectivamente reglas de asociación entre líneas espectrales que resultaron tener mayor relevancia sobre el conjunto de datos bajo distintas criterios, al aplicarse sobre conjuntos de datos reales obtenidos a partir de observaciones en el espectro visible. Se logró observar conjuntos de especies que están presentes con mayor frecuencia, y las reglas que estos generan; con sus medidas estadísticas correspondientes. Además se logró inferir conclusiones sobre el desempeño de los algoritmos implementados al comparar su eficiencia sobre datos reales con su eficiencia sobre datos simulados.

Queda para desarrollo a futuro una implementación más general del procedimiento para así aplicar los algoritmos a datos obtenidos en otras bandas de frecuencia, como por ejemplo, las observaciones radioastronómicas de ALMA; que por sus caracterísiticas, promete un mayor número de datos sobre los cuales obtener reglas de asociación para líneas espectrales.

\end{intro}