\chapter{Marco Teórico}

% Conceptos involucrados
% - Nuevas tecnologías consideradas
% - Metodologías de desarrollo
% - Algoritmos existentes
% - Arquitecturas estándar
% - Descripción de las soluciones existentes
% Concentra la mayor parte de las citas
% Alquien que conoce del tema podría saltearse este capítulo e igual comprender la memoria

\section{Antecedentes Astronómicos}

\subsection{Espectroscopía Astronómica}

En los comienzos del siglo XIX, los astrónomos comenzaron a realizar medidas que, por primera vez, revelaron con exactitud cuán lejanas se encuentran de la tierra incluso las estrellas más cercanas. Y, al igual que entonces, actualmente sigue siendo técnicamente imposible, con la tecnología que contamos, viajar a las cercanías de estos objetos estelares. Sin embargo, hoy en día es bastante bien conocida la composición química de las estrellas y del material difuso presente en los vastos espacios que las separan. El estudio de los espectros de los objetos astronómicos, o \emph{espectroscopía astronómica}, es lo que ha hecho esto posible.

En el año 1814, el científico Joseph von Fraunhofer (1787 - 1826), mediante el uso de prismas de alta calidad construidos por él mismo, logró difractar un rayo de luz solar y proyectarlo hacia un muro blanco. Además de los colores característicos del arcoíris, observados de esta manera desde los tiempos de Newton, vio en la proyección resultante muchas líneas oscuras. Procedió, luego, a catalogar meticulosamente la longitud de onda exacta de cada una de estas líneas, que hasta el día de hoy se conocen como líneas de Fraunhofer, y asignó letras a las más notorias. De esta forma, Fraunhofer registró el primer espectro astronómico de alta resolución.

Ahora bien, él no sabía cuál era la causa de que estas líneas oscuras estuvieran presentes en el espectro. Sin embargo, posteriormente procedió a realizar el mismo experimento, pero esta vez utilizando un rayo de luz proveniente de la estrella roja cercana Betelgeuse, y observó que el patrón de líneas oscuras cambiaba considerablemente. Fraunhofer concluyó correctamente que estas se encuentran de cierta forma relacionadas con la composición del objeto observado. En efecto, algunas de las líneas observadas por Fraunhofer se deben a las especies (e.g átomos, iones, moléculas) que componen la atmósfera terrestre.

Sin embargo, el gran paso en la comprensión general de las observaciones de Fraunhofer llegó a mediados del siglo XIX de la mano del trabajo de los científicos Gustav Kirchhoff (1824 - 1887) y Robert Bunsen (1811 - 1899), quienes estudiaron el color de la luz emitida al poner distintos metales en llamas. Al hacer esto, descubrieron que, en ciertos casos, la longitud de onda de la luz emitida coincidía exactamente con las líneas observadas por Fraunhofer. Estos experimentos demostraron que las líneas de Fraunhofer son una consecuencia directa de la composición atómica del sol.

En el siglo XX se llegó a comprender de manera más profunda la razón de la existencia de estas líneas, denominadas \textit{líneas espectrales}, gracias a la revolución que significó la llegada de la mecánica cuántica. Los desarrollos en materia de espectroscopía han estado, desde entonces, estrechamente ligados a los de aquel campo de la física.

Hoy en día, esencialmente toda la información con la que se cuenta sobre objetos astronómicos que residen fuera del sistema solar se ha obtenido mediante el estudio de la radiación electromagnética que estos emiten. Esta radiación contiene mucha información detallada, la cual puede ser obtenida solo mediante un análisis cuidadoso. En términos generales, se puede clasificar la información obtenida a partir de esta radiación según la resolución espectral; esto es, el grado de sensibilidad a las distintas longitudes de onda utilizado para realizar la observación.

Por ejemplo, cuando se observa el cielo de noche directamente con la vista, la mayoría de los objetos astronómicos se ven blancos. La luz blanca es en realidad luz que consta de muchas longitudes de onda y que no ha sido descompuesta en sus distintos colores. Observando esta luz blanca es posible obtener las posiciones de los objetos en el cielo nocturno, construir mapas de estrellas y galaxias, y registrar el movimiento de cuerpos celestes; como se ha hecho durante siglos hasta nuestros días.

Si se observa cuidadosamente ciertos objetos, tales como los planetas Marte o Júpiter, o estrellas tales como Betelgeuse, se puede apreciar que estos objetos tienden a tener un cierto color. Basta utilizar instrumentos de bajo poder resolutivo para separar la luz que llega desde estos objetos a la tierra en colores de amplio espectro. A su vez, el observar estos colores entrega información sobre la temperatura del objeto. Por ejemplo, las estrellas azules poseen mayor temperatura que las rojas. Objetos que emiten rayos x, como la corona solar, son muy calientes, mientras que objetos fríos emitirán radiación en longitudes de onda mayores; por ejemplo, en forma de ondas de radio.

La única forma de obtener información astrofísica detallada de objetos del cielo es mediante observaciones de alta resolución que involucren el detectar la intensidad de la radiación recibida en función de las longitudes de onda que la componen. Esto se lleva a cabo con equipos de alto poder resolutivo y sensibilidad. Dos ejemplos de estos son, el telescopio óptico SDSS que se encuentra en el Apache Point Observatory (APO, ubicado en Nuevo México, Estados Unidos) y con el cual se lleva a cabo el \textit{Sloan Digital Sky Survey (SDSS)}; y, en mayor medida, el interferómetro radioastronómico \textit{Atacama Large Millimeter/submillimeter Array (ALMA)} ubicado en el norte de Chile.

Observaciones llevados a cabo con estos equipos de alta resolución permiten obtener, no solamente la posición central de una línea dentro del espectro, sino también su forma. A partir de esta información, y con un conocimiento previo de física atómica y molecular, puede extraerse valioso conocimiento sobre muchas de las propiedades del objeto y de su composición. Dado que existe una relación directa entre los parámetros físicos subyacentes y la información astronómica que se puede extraer a partir de los espectros, es posible utilizar datos generados a partir de observaciones experimentales en un laboratorio y compararlos con las líneas del espectro obtenidos de un objeto del cielo. Mediante este procedimiento se puede inferir propiedades del objeto, tales como su composición química, su temperatura, la abundancia de las especies que lo componen y que se encuentran emitiendo radiación, el movimiento de las especies y del objeto en sí, la presión y densidad local, el campo magnético presente, entre otros.

En síntesis, si se conoce esta información a partir de los datos de laboratorio, entonces una vez que se detecta un conjunto de líneas espectrales y se sabe la longitud de onda a la cual fueron emitidas es posible conocer a qué especies corresponden, y por tanto saber la composición química del objeto observado.

\subsection{Sloan Digital Sky Survey (SDSS)}

El \textit{Sloan Digital Sky Survey (SDSS)} es un proyecto de inspección y estudio del espacio llevado a cabo mediante el uso de un telescopio óptico ubicado en el observatorio Apache Point (APO), Nuevo México, Estados Unidos. La recolección de datos comenzó en el año 2000, y las imágenes finales de los datos publicados cubren un 35\% del cielo, con observaciones fotométricas de 500 millones de objetos y espectros de radiación electromagnética de 1 millón de objetos.

El telescopio hace uso de la rotación terrestre para capturar pequeñas franjas del cielo, las cuales son registradas en un circuito integrado llamado dispositivo de carga acoplada o \textit{charge-coupled device (CCD)} que captura las imágenes y permite transmitirlas y almacenarlas en formato digital.

Utilizando los datos obtenidos de esta forma, se seleccionan objetos del cielo para su análisis espectroscópico. El espectrógrafo opera mediante asignar una fibra óptica individual a cada objeto que se desea observar y fijándola en su posición correspondiente a través de un agujero en una placa de aluminio. Cada agujero se ubica específicamente para el objeto deseado, por lo tanto, distintas áreas del cielo con distintos objetos requieren distintas placas de aluminio. El espectrógrafo de uso actual es capaz de registrar 1000 espectros a la vez. Cada noche se utilizan entre 6 a 9 placas para registrar espectros.

Los datos de SDSS se hacen disponibles mediante publicaciones regulares o \textit{data releases} a través de internet. La última publicación llevada a cabo fue la correspondiente al data release 10 (DR10), con fecha de julio del 2013. Los datos de todos los data releases se encuentran en un servidor \textit{Microsoft SQL Server} y pueden accederse mediante diversas interfaces o APIs presentes en el sitio web de SDSS. En particular, existe una interfaz web llamada \textit{CasJobs} que permite realizar consultas en lenguaje \textit{SQL} a un servidor que encola la petición, la ejecuta y guarda los resultados en una base de datos asignada al usuario.

\subsection{Atacama Large Millimeter/submillimeter Array (ALMA)}

El \textit{Atacama Large Millimeter/submillimeter Array (ALMA)} es un interferómetro astronómico de radiotelescopios ubicados en el desierto de Atacama, en el norte de Chile. Es parte de un proyecto llevado a cabo mediante una asociación de organizaciones de Norteamérica, Europa y el este de Asia. Comenzó sus observaciones científicas en la segunda mitad del año 2011. Se encuentra completamente operacional desde marzo del año 2013; y es el mayor y más caro radiotelescopio construido hasta la fecha.

ALMA realiza observaciones captando radiación electromagnética proveniente del espacio en bandas milimétricas y submilimétricas en sus longitudes de onda, que corresponden a ondas de radio. Debido a que en condiciones normales la humedad del ambiente y del cielo absorbe gran parte de este tipo de radiación, es crucial para el funcionamiento de los telescopios el estar ubicados en uno de los lugares más secos del mundo, el llano de Chajnantor en el desierto de Atacama, a más de 5000 metros de altura. 

Gran parte de los datos obtenidos desde ALMA son guardados en estructuras de datos llamadas cubos de datos tipo ALMA (o \textit{ALMA Data Cubes}), que contienen información de distintos puntos de observación del cielo a distintas frecuencias. 
Los espectros de frecuencia son una forma de representar la intensidad de la radiación electromagnética, recibida desde un punto del espacio, en un cierto rango de frecuencias. Estos contienen puntos altos de intensidad en ciertas frecuencias en las cuales se sabe que una cierta molécula conocida efectúa una transición cuántica. Por lo tanto, mediante reconocer e identificar estos puntos altos, o \textit{peaks}, se puede saber las transiciones moleculares que ocurrieron en el objeto del que proviene la radiación electromagnética. Como, a su vez, se sabe de antemano a qué frecuencia específicamente se realizan las transiciones cuánticas de moléculas conocidas, puede inferirse cuáles son las moléculas presentes en el objeto de origen.

A su vez, los cubos de datos tipo ALMA, como estructura de datos, contienen valores indexados en tres coordenadas. Dos de las coordenadas son espaciales, y corresponden al equivalente a una imagen normal de dos dimensiones, en el sentido que describen puntos del cielo (o del espacio observable desde la tierra). La tercera coordenada corresponde al rango de frecuencias en el que se está detectando radiación electromagnética. Por lo tanto, si se fijan las dos coordenadas espaciales (se fija un punto en el espacio) y se extraen todos los valores en la tercera coordenada de aquel punto, se obtiene el espectro de frecuencias observado en ese punto del espacio.

Los cubos de datos tipo ALMA, por tanto, contienen información de los espectros de frecuencia observados en todos los puntos de un sector dado del espacio. Todos los espectros presentes en un cubo se encuentran en un mismo rango de frecuencia. Sin embargo, distintos cubos de datos pueden tener observaciones hechas en distintos rangos de frecuencia entre sí.

A partir de ALMA se generan enormes cantidades de datos, los cuales, debido a su gran tamaño,  necesariamente deben procesarse por parte de sistemas automatizados de extracción y análisis con el fin de facilitar a los investigadores el extraer información útil a partir de estos. La mayoría de estas herramientas se encuentran dentro de las áreas de investigación de la minería de datos y Machine Learning; disciplinas de la computación que han tenido un gran auge en el último tiempo.

\section{Reglas de asociación}

El aprendizaje mediante reglas de asociación, o \textit{Association Rule learning (ARL)}, es sin lugar a dudas uno de los métodos más populares y mejor estudiados dentro de la minería de datos. Basta para ello ver que el artículo seminal de Agrawal et al.\cite{agrawal1993mining}, donde se sentaron las bases de la teoría subyacente, es uno de los más citados del área; según el catálogo y herramienta de búsqueda de publicaciones científicas \textit{Google Scholar}.

La motivación principal de ARL en su concepción fue el encontrar relaciones lógicas entre los artículos adquiridos por usuarios en puntos de venta del tipo \textit{"Si un cliente compra los artículos A y B, entonces es muy probable que también compre el artículo C"}. Sin embargo, la teoría de fondo que se desarrolló con el tiempo tiene una gran cantidad de aplicaciones en los más diversos ámbitos.

\subsection{Definición formal}

Sea $\mathcal{I} = \{i_1,\,i_2,\,i_3,\,\ldots ,\,i_m\}$ un universo de ítemes posibles. Se denomina, entonces a un conjunto $X \subseteq \mathcal{I}$ como \textit{conjunto de ítemes} o \textit{itemset}. Se tiene, además un conjunto de transacciones $\mathcal{T} = \{T_1,\,T_2,\,\ldots,\,T_n\}$, donde $T_i \subseteq \mathcal{I}, \; \forall i \in {[1,n]}$. Dados un conjunto de ítemes $X$ y una transacción $T_i$, se dice que la trasacción $T_i$ \textit{satisface} $X$ si y solo si $X \subseteq I_i$.

Una \textit{regla de asociación} es, entonces, una relación (más específicamente, una implicancia) entre dos conjuntos de la forma $X \Rightarrow Y$, donde $X \subset \mathcal{I}$, $Y \subset \mathcal{I}$, y $X \cap Y = \emptyset$. A $X$ se denomina el \textit{antecedente} de la regla y a $Y$ se denomina el \textit{consecuente} de la regla.

Existen, también, una serie de medidas para cuantificar la relevancia de una regla de asociación. A continuación se define algunas de ellas.

El \textit{soporte} de un conjunto de ítemes $X$, o $\mathit{supp}(X)$, se define como $$\mathit{supp}(X) = \frac{|\mathcal{T}_X|}{|\mathcal{T}|} \quad \text{, tal que} \quad \mathcal{T}_X = \{T \in \mathcal{T} \, : \, X \subset T \}\text{,}$$ vale decir, corresponde a la fracción del total de transacciones en la que está presente el conjunto.

A su vez, el soporte de una regla de asociación $X \Rightarrow Y$, o $\mathit{supp}(X \Rightarrow Y)$, se define como $$\mathit{supp}(X \Rightarrow Y) = \mathit{supp}(X \cup Y)\text{,}$$ vale decir, corresponde a la fracción del total de transacciones en las cuales está presente tanto el antecedente como el consecuente de la regla simultáneamente\footnote{Debe tenerse en mente que la expresión $mathit{supp}(X \cup Y)$ indica la fracción del total de transacciones en las cuales está presente \textbf{tanto} el antecedente como el consecuente de la regla \textbf{simultáneamente}, y \textbf{no} de aquellas en las cuales está presente el antecedente \textbf{o} el consecuente. El argumento del soporte $\mathit{supp}$ es un conjunto de ``pre-condiciones", y, por lo tanto, se vuelve más restrictivo en la medida que su cardinalidad aumenta.}.

La \textit{confianza} de una regla de asociación $X \Rightarrow Y$, denotada por $\mathit{conf}(X \Rightarrow Y)$, se define como $$\mathit{conf}(X \Rightarrow Y) = \frac{\mathit{supp}(X \cup Y)}{\mathit{supp}(X)}\text{,}$$ es decir, indica en qué fracción de las transacciones en las cuales está presente el antecedente la regla se cumple (i.e. está presente también el consecuente de la regla). Debido al uso frecuente de esta medida de relevancia, resulta usual el expresar una regla de asociación mediante la notación $$X \Rightarrow Y \, \Big| \, c$$ donde $c = \mathit{conf}(X \Rightarrow Y)$.

El \textit{lift} de una regla de asociación $X \Rightarrow Y$, denotado por $\mathit{lift}(X \Rightarrow Y)$, se define como $$\mathit{lift}(X \Rightarrow Y) = \frac{\mathit{conf}(X \Rightarrow Y)}{\mathit{supp}(Y)} = \frac{\mathit{supp}(X \cup Y)}{\mathit{supp}(X) \times \mathit{supp}(Y)}\text{.}$$ La intuición detrás del concepto de lift tiene lugar al interpretar las medidas descritas anteriormente desde un punto de vista probabilístico. Tomando el conjunto $\mathcal{T}$ como un universo de posibles resultados, o espacio muestral, se tiene que $$\mathit{supp}(X) = P(X) \quad \text{y} \quad \mathit{conf}(X \Rightarrow Y) = P(Y| X)\text{.}$$ Desde este punto de vista, la medida de lift indica qué tan bien la presencia del antecedente de una regla lograría predecir la presencia del consecuente. Por lo tanto, si la presencia del antecedente y del consecuente en una transacción cualquiera son eventos estadísticamente independientes (i.e. la ocurrencia de uno no afecta la probabilidad de que el otro ocurra), se tendrá que $\mathit{lift}(X \Rightarrow Y) = 1$; y este valor irá variando en la medida que ambos eventos sean más dependientes entre sí.

Por ejemplo, supongamos que se tiene el siguiente conjunto de transacciones

\begin{tabular}{l l}
\textbf{TID} & \textbf{Items} \\
1 & $a$, $c$ \\
2 & $a$, $d$ \\
3 & $b$, $c$ \\
4 & $b$, $d$ \\
\end{tabular}

donde \textit{TID} es el número identificador de la transacción. Luego, para este caso, se tiene que $$\mathit{lift}(\{a\} \Rightarrow \{c\}) = \frac{\mathit{supp}(\{a\} \cup \{c\})}{\mathit{supp}(\{a\}) \times \mathit{supp}(\{c\})} = \frac{1/4}{1/2 \times 1/2} = 1\text{,}$$ lo cual indica que la que la ocurrencia de que una transacción cualquiera satisfaga $\{a\}$ es estadísticamente independiente de que una transacción cualquiera satisfaga $\{b\}$.

En cambio, en el siguiente conjunto de transacciones

\begin{tabular}{l l}
\textbf{TID} & \textbf{Items} \\
1 & $a$, $c$ \\
2 & $a$, $d$ \\
3 & $b$, $c$ \\
4 & $b$, $c$ \\
\end{tabular}

se tiene que $$\mathit{lift}(\{a\} \Rightarrow \{c\}) = \frac{\mathit{supp}(\{a\} \cup \{c\})}{\mathit{supp}(\{a\}) \times \mathit{supp}(\{c\})} = \frac{1/4}{1/2 \times 3/4} = 2/3 < 1\text{,}$$ lo cual quiere decir que hay una mayor razón de transacciones que satisfacen $\{c\}$ dentro del total de transacciones que dentro del conjunto de transacciones que satisfacen $\{a\}$.

Finalmente, en el conjunto de transacciones

\begin{tabular}{l l}
\textbf{TID} & \textbf{Items} \\
1 & $a$, $c$ \\
2 & $a$, $d$ \\
3 & $b$, $d$ \\
4 & $b$, $d$ \\
\end{tabular}

se cumple que $$\mathit{lift}(\{a\} \Rightarrow \{c\}) = \frac{\mathit{supp}(\{a\} \cup \{c\})}{\mathit{supp}(\{a\}) \times \mathit{supp}(\{c\})} = \frac{1/4}{1/2 \times 1/4} = 2 > 1\text{,}$$ lo cual indica que hay una mayor razón de transacciones que satisfacen $\{c\}$ dentro del conjunto de transacciones que satisfacen $\{a\}$ que dentro del total de transacciones.

\subsection{Algoritmos, implementaciones y aplicaciones}

En el mismo artículo seminal de ARL por Agrawal et al.\cite{agrawal1993mining}, se presentó el algoritmo \textit{Apriori}. Este algoritmo hace uso de las propiedades de clausura descendiente de la frecuencia de los conjuntos con respecto a sus subconjuntos con el fin de optimizar el proceso de generación de conjuntos de ítemes frecuentes.

Posteriormente, Agrawal et al. presentaron el algoritmo \textit{AprioriTid}, cuyas mejores características fueron combinadas con el algoritmo \textit{Apriori} para crear el algoritmo \textit{AprioriHybrid}, de orden de complejidad lineal en el número de transacciones\cite{agrawal1994fast}. Luego se han realizado más desarrollos en ARL orientado a transacciones secuenciales de clientes de puntos de ventas\cite{agrawal1995mining}.

Savasere et al. introdujeron el algoritmo \textit{Partition}\cite{savasere1995efficient} con el fin de extraer reglas de asociación en base de datos, el cual presenta reducciones en las operaciones de la CPU y de entrada/salida, y que además facilita la paralelización. Posteriormente se creó el algoritmo \textit{Dynamic Itemset Counting (DIC)}\cite{brin1997dynamic}, que realiza menos lecturas sobre los datos que los algoritmos previos, y que utiliza la métrica de \textit{Convicción} a la hora de generar reglas de asociación. Luego, Park et al. presentaron un algoritmo que hace uso de funciones de Hashing con el fin de generar reglas candidatas\cite{park1995effective}. Se han realizado, también, adaptaciones de los algoritmos previos con el fin de realizar ARL en datos de tipo cuantitativo\cite{srikant1996mining}.

Esfuerzos posteriores se han realizado con el fin de profundizar en los fundamentos teóricos subyacentes en ARL (e.g. definiendo el conjunto de posibles ítemes como una estructura algebráica llamada \textit{retículo})\cite{zaki1998theoretical}, y con el fin de extender la noción de reglas de asociación a correlaciones\cite{brin1997beyond}.

Más recientemente, Han et al. introdujeron el uso de una estructura de datos llamada \textit{Frequent Pattern Tree}\cite{han2004mining} en la extracción de reglas de asociación a partir de conjuntos de transacciones. Luego de esto, se han hecho numerosas implementaciones y optimizaciones a los algoritmos más utilizados en ARL, como, por ejemplo, el algoritmo Apriori\cite{bodon2010fast}; así como implementaciones que facilitan el mantener la privacidad de cada una de las fuentes de datos que participan en el proceso\cite{evfimievski2004privacy}.

Desde su concepción, el método de ARL ha sido aplicado en numerosas áreas, tales como la detección de intrusiones\cite{lee2000framework} y anomalías\cite{patcha2007overview}\cite{chandola2009anomaly}, educación\cite{romero2007educational}\cite{romero2008data}, química\cite{dehaspe1998finding}, privacidad de datos\cite{ghinita2008private}, búsqueda en la web\cite{ferragina2008personalized}, tráfico en redes\cite{estan2003automatically}, computación social\cite{li2008tag}, búsqueda semántica\cite{cohen2007associative}, biología\cite{kramer2001molecular}\cite{carmona2007genecodis}, salud\cite{karabatak2009expert}\cite{chaves2011efficient}, medios de comunicación\cite{davidson2010youtube}\cite{kobilarov2009media}, y la investigación forense\cite{iqbal2013unified}. Junto con esto, se han realizado numerosas investigaciones sobre el estado actual de ARL y sus posibles desarrollos a futuro dentro del marco de métodos automatizados de generación de conocimiento\cite{han2007frequent}.

Si bien existen numerosos esfuerzos por utilizar minería de datos y Machine Learning en diversos ámbitos de la astronomía (en particular, en detección, clasificación y caracterización de líneas moleculares en espectros de emisión\cite{vskoda2011searching}), hasta la fecha no se ha propuesto abiertamente el uso de ARL sobre datos extraídos de espectros de frecuencia.

Sin embargo, se han realizado avances en ampliar los conceptos subyacentes en ARL con el fin de aplicar el método en campos más diversos\cite{brin1997beyond}. Específicamente, una rama de investigación ha desarrollado lo que se denomina \textit{Weighted Association Rule Learning}\cite{wang2000efficient}\cite{cai1998mining}. Este método permite asociar medidas de interés arbitrario a priori a ciertos conjuntos de datos. Si bien esto hace que se pierdan propiedades de clausura que son útiles a la hora de generar algoritmos eficientes, también permite trabajar con distintos conjuntos de transacciones sin que las reglas generadas estos dependan exclusivamente de su soporte u otras medidas estándar.

