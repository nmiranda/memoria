\chapter{Especificación del Problema}

% Descripción detallada del problema a resolver
% Se discute la relevancia de contar con una solución
% Se especifica en todo detalle los requisitos de la solución a construir
% Características de calidad de la solución deseada
% Es deseable también establecer criterios de aceptación de una solución al problema

\section{Descripción del problema}

\todo[inline]{Explicar que las líneas ya han sido detectadas. Como? Mencionar o citar el trabajo de Karim Pichara y Andrés Riveros?}

Supóngase que se cuenta con conjuntos de espectros, y que cada uno de ellos posee todas sus líneas espectrales correctamente detectadas y, por lo tanto, se conoce su posición en el espectro. En la práctica eso puede ser muy difícil de lograr, sobre todo en circunstancias donde pueden existir en principio una alta cantidad de líneas espectrales y estas pueden interferir unas con otras en la señal final, lo que se conoce como \textit{blending}.\todo[inline]{No sé si poner lo anterior acá o en el capítulo de marco teórico} Por lo tanto, para efectos de lo que sigue, basta con asumir que existe la posibilidad que no todas las líneas hayan sido detectadas. Pero es importante que las que sí fueron detectadas, lo hayan sido con una seguridad suficiente y que se sepa de manera adecuada su posición.\todo[inline]{Explicar cómo se llevó a cabo la detección en términos más rigurosos. Cómo?}

Teniendo estos conjuntos de espectros con sus respectivas líneas detectadas se desea... 

\todo[inline]{Definir el problema de forma general, quizás con lineamientos de objetivos, que no involucre directamente como solución el uso de reglas de asociación}

\section{Requisitos de la solución y casos de uso}

\todo[inline]{Definir los casos de uso}

\missingfigure{Diagrama de casos de uso}