\documentclass[upright, contnum]{umemoria}

\depto{DEPARTAMENTO DE CIENCIAS DE LA COMPUTACI\'ON}
\author{NICOL\'AS MART\'IN MIRANDA CASTILLO}
\title{REGLAS DE ASOCIACI\'ON PARA L\'INEAS ESPECTRALES}
\auspicio{el D\'ECIMO NOVENO 
CONCURSO DE PROYECTOS DE INVESTIGACI\'ON Y DESARROLLO FONDEF 2011, 
PROYECTO FONDEF D11I1060, y el CENTRO DE MODELAMIENTO MATEM\'ATICO DE LA UNIVERSIDAD DE CHILE (CMM)}
\date{DICIEMBRE 2014}
\guia{GUILLERMO CABRERA VIVES}
\carrera{INGENIERO CIVIL EN COMPUTACI\'ON}
\memoria{MEMORIA PARA OPTAR AL T\'ITULO DE}
\comision{GONZALO NAVARRO BADINO}{PABLO GUERRERO P\'EREZ}{\ }


%\usepackage[utf8]{inputenc}
%\usepackage[T1]{fontenc}
%\usepackage[latin1]{inputenc}

\usepackage{lipsum}
\usepackage{verbatim}
\usepackage{fontspec}
\usepackage{xunicode}
\usepackage{setspace}
\usepackage{todonotes}
\usepackage{listings}
\usepackage[lined,boxed,commentsnumbered,ruled,vlined]{algorithm2e}
\usepackage{longtable}
\usepackage{hyperref}

\begin{document}

\frontmatter
\maketitle

\begin{abstract}
En el presente trabajo se llevó a cabo la implementación de algoritmos de reglas de asociación con la finalidad de inferir relaciones lógicas existentes en grandes cantidades de datos. En particular, se busca aplicar a conjuntos de líneas espectrales extraídas a partir de datos de observaciones astronómicas, para así obtener información de las relaciones existentes entre ellas bajo distintas medidas de interés y relevancia estadística.

Para ello se utilizó algoritmos de Aprendizaje de Reglas de asociación, o \textit{Association Rule Learning (ARL)}; en particular los algoritmos \textit{Apriori} y \textit{FP-Growth}. La aplicación final permite al usuario observar las reglas obtenidas bajo requerimientos mínimos de \textit{soporte} y \textit{confianza} de ellas, ordenarlas según estas dos medidas junto con su \textit{lift}, y mostrar las que posean un cierto elemento en particular en su antecedente, consecuente o en ambos.

La aplicación se probó sobre datos de observaciones ópticas obtenidas del \textit{Sloan Digital Sky Survey (SDSS)}, previo un pre-procesamiento adecuado de estos, y se espera a futuro poder realizar el proceso de ARL a partir datos en otras frecuencias del espectro electromagnético; como por ejemplo, los datos radioastronómicos del \textit{Atacama Large Millimeter/submillimeter Array (ALMA)}.
\end{abstract}

\begin{dedicatoria}
A mi padre.
\end{dedicatoria}

\begin{thanks}
A propósito del trabajo que aquí se presenta, no quisiera dejar pasar esta ocasión sin agradecer a Guillermo Cabrera, mi guía a lo largo de este proyecto, por su paciencia, instrucción y excelente disposición a la hora de permitirme ver y aprender muchas cosas nuevas. Muchas gracias, también, al profesor Diego Mardones. Sin su asesoramiento en materias científicas que incluyen (pero no se reducen solo) a la astronomía, y su constante ayuda en general, este trabajo no habría sido posible.

En términos más personales, profundos y generales, a mi familia. A Liliana, mi madre, por el cariño sin medidas ni reservas que siempre me ha brindado. A Rocío, mi hermana, por su alegría contagiosa y optimismo, que en más de una ocasión me han sacado adelante. Y, por supuesto, a Sergio, mi padre, por su apoyo incondicional, por su ejemplo, compañía y enseñanzas invaluables sobre nunca darse por vencido, sin dejar de disfrutar del día a día. No estaría aquí de no ser por ustedes.

A mis amigos de siempre y de ahora, en recuerdo pero sobre todo en presencia. Gracias por haber compartido conmigo tantos buenos momentos, risas, ideas, conversaciones, y por estar aun ahí para mí, a pesar de lo divergentes que son a veces los senderos de la vida.

A todos los creadores, escritores, profesores, artistas, personas comunes y anónimas que, mediante sus obras y ejemplos, me han enseñado el valor de pensar por uno mismo, ver más allá de lo evidente, sorprenderse con la realidad, imaginar sin temores y apreciar el mundo del que todos somos parte.

A todos ustedes, muchas gracias.
\end{thanks}

\cleardoublepage
\begin{spacing}{1}
\tableofcontents
%\cleardoublepage
%\listoftables
%\cleardoublepage
\listoffigures
\end{spacing}

\mainmatter

%\section{Introducción}
\begin{intro}

% Resume todo pero en más detalle
% - Contexto
% - Problema
% - Relevancia / Motivación para encontrar una solución
% - Alternativas analizadas, alternativa escogida
% - Descripción general de la solución
% - Resultados de la solución implementada para resolver el problema

\vspace{1em}
\hfill{}
\begin{minipage}{9cm}{
\begin{spacing}{0.9}
\small
\noindent
\textit{[...] we may in time ascertain the mean temperature of heavenly bodies, but I regard this order of facts as for ever excluded from our recognition. We can never learn their internal constitution [...]}
\end{spacing}
\vspace{1em}
\hfill{}{Auguste Comte, \textit{Astronomy, Ch. I: General View}, 1835}
}
\vspace{2em}
\end{minipage}

\section*{Contexto y motivación}
\addcontentsline{toc}{section}{Contexto y motivación}

En los últimos tiempos, y en gran parte debido al explosivo desarrollo tecnológico, han surgido numerosos campos en los cuales se ha requerido el uso de procesamiento masivo de datos e inteligencia computacional con el fin de automatizar y auxiliar el proceso de generación de nuevo conocimiento. La astronomía es, sin lugar a dudas, uno de ellos. Esto se debe, en parte, al explosivo desarrollo de nuevas tecnologías que ponen al alcance de la comunidad científica una cantidad nunca antes vista de datos; los cuales contienen abundante información sobre el universo, su composición, estructura, origen y destino.

Un claro ejemplo de esto lo constituye el \textit{Atacama Large Millimiter/sub-millimiter Array (ALMA)} \cite{wootten2009atacama}, un interferómetro radio-astronómico que consiste en un arreglo de 66 antenas que observan el espacio en las bandas milimétricas y submilimétricas del espectro electromagnético. Ubicado en el desierto de Atacama, en el norte de Chile, es parte de uno de los proyectos científicos más importantes del último tiempo; en el cual se ha hecho uso de tecnologías de punta por parte de investigadores, ingenieros y técnicos expertos en diversas áreas del conocimiento, tales como la astronomía, la computación científica y de alto rendimiento, la electrónica, entre otros.

La tecnología involucrada en el proyecto \textit{ALMA} ha permitido, entre otras cosas, obtener datos de alta resolución provenientes de distintas fuentes u objetos del espacio observable desde la tierra. La radiación electromagnética emitida por estos objetos, en bandas de frecuencia de radio, son captadas por el arreglo de antenas y posteriormente procesadas por equipos de alta capacidad con el fin de obtener los espectros electromagnéticos correspondientes. Estos, a su vez pueden ser analizados directamente o utilizarse para generar imágenes de alta calidad.

Parte principal de la importancia de estos espectros de radiación electromagnética es que dan información valiosa sobre la composición química de los objetos de los que esta proviene. Esto se debe a que los átomos que componen estos objetos emiten o absorben una mayor cantidad de energía en frecuencias muy específicas. Por lo tanto, un espectro en particular tendrá rangos estrechos de mayor o menor intensidad en ciertas frecuencias dependiendo de los elementos químicos de los que está compuesto el objeto del que proviene.

La detección de lineas espectrales es un problema de interés en sí, y que puede llegar a ser muy complejo dependiendo de en qué bandas de frecuencia se esté trabajando. Sin embargo, se puede seguir obteniendo información valiosa de los objetos observados a partir de estas líneas ya detectadas. Esto incluye potencialmente respuestas a preguntas como: ¿de qué forma se relacionan ciertos tipos de líneas entre sí? ¿Existe una mayor correlación de presencia de líneas de ciertos isótopos o moléculas en particular? ¿Hay una mayor presencia de líneas de ciertas especies en algunos objetos que en otros? ¿Qué nos dice esto de la composición de los objetos y de su química subyacente?

Existe, en el dominio de la minería de datos, el concepto de \textit{reglas de asociación}; las cuales corresponden a asociaciones lógicas entre conjuntos de elementos o ítems. Si bien estos inicialmente fueron concebidos con el fin de resolver problemas pertenecientes al ámbito del comercio y las ventas, hoy en día son aplicados en los más diversos contextos. Es, por lo tanto, una de las finalidades principales de este trabajo, el mostrar que la espectroscopía astronómica no es la excepción, y que es posible extraer reglas de asociación entre líneas espectrales obtenidas a partir de observaciones de objetos del espacio.

Se espera que el generar de manera automática reglas de asociación entre líneas moleculares facilite, a futuro, el análisis de la naturaleza de los objetos observados, y las caracterísicas de su composición química; al averiguar cómo se relacionan entre sí los elementos, átomos, moléculas e isótopos presentes en las sustancias que los componen. Esto, sobre todo, en vista de que hoy en día el volumen de datos generados a partir de observaciones astronómicas no deja de aumentar.

\section*{Objetivos}
\addcontentsline{toc}{section}{Objetivos}

\subsection*{Objetivo General}
\addcontentsline{toc}{subsection}{Objetivo General}

\begin{itemize}
	\item Implementar un sistema de aprendizaje de reglas de asociación, o \textit{Association Rule Learning (ARL)}, que permita obtener relaciones lógicas entre líneas espectrales presentes dentro de un conjunto de datos de espectroscopía astronómica.
\end{itemize}

\subsection*{Objetivos Específicos}
\addcontentsline{toc}{subsection}{Objetivos Específicos}

\begin{itemize}
  \item Implementar un sistema de ARL genérico que permita aplicarse a datos provenientes de diversos orígenes.
  \item Obtener reglas de asociación entre líneas espectrales obtenidas a partir de datos reales.
  \item Visualizar las reglas de asociación, presentes en el conjunto de datos, que sean de mayor interés según medidas estadísticas. 
  \item Filtrar las reglas de asociación encontradas en un conjunto de datos de espectroscopía astronómica según las líneas que las componen.
\end{itemize}

\section*{Descripción de la solución}
\addcontentsline{toc}{section}{Descripción de la solución}

Si bien existen diversas técnicas de clasificación y caracterización de puntos en un espacio multidimensional (en nuestro caso objetos descritos por parámetros), para resolver las preguntas anteriores se requiere más bien de una herramienta que permita encontrar relaciones explícitas entre los parámetros en sí, y que permita asignar medidas de relevancia estadística a estas relaciones.

Para ello se planteó el uso de \textit{Association Rule Learning (ARL)}, o Aprendizaje de Reglas de Asociación, como una herramienta que puede dar respuesta directa a algunas de las interrogantes mencionadas anteriormente, y ayudar a obtener información clave para el proceso de utilizar otras técnicas en el largo plazo.

El Aprendizaje de Reglas de Asociación como técnica se ubica dentro del área de la minería de base de datos, y su concepción original fue el ser aplicada a sistemas de puntos de venta con el fin de encontrar las relaciones más comunes entre artículos comprados por los clientes. Sin embargo, con el tiempo se ha convertido en una de las herramientas más utilizadas de su área, en una diversa gama de contextos.

En el presente trabajo se llevó a cabo el uso de esta técnica con el fin de encontrar relaciones comunes entre líneas espectrales a través de distintos espectros de frecuencia. Ahora bien, la naturaleza innata de estos es más bien contínua y las líneas en sí mismas poseen diversos parámetros que las caracterizan. Por lo tanto, este caso dista mucho de la binaridad del problema original para el cual se pensó ARL. Sin embargo, como se muestra a lo largo de este informe, si se asume que se realizó con anterioridad un buen trabajo de detección de líneas y se efectúa un pre-procesamiento adecuado de los datos, el algoritmo de ARL arroja resultados que están en concordancia con la química subyecente.

En particular, se utilizó una implementación de dos de los algoritmos más utilizados de ARL: \textit{Apriori} y \textit{FP-Growth}. Luego, se obtuvo una base de datos de líneas espectrales ya detectadas (pero no necesariamente asociadas a alguna especie [átomo, isótopo, etc.]) correspondientes a observaciones del \textit{Sloan Digital Sky Survey (SDSS)}, un sondeo espectroscópico del espacio realizado con un telescopio óptico. Sobre este conjunto de datos se procedió a realizar un pre-procesamiento que, entre otros, consta de filtrar las líneas según su brillo o razón señal a ruido. Luego, se efectuaron particiones según las características de los objetos de procedencia (como tipo de objeto o estructura estelar, cercanía, etc.). Finalmente, sobre estas se procedió a aplicar los algoritmos de ARL.

Los resultados obtenidos fueron efectivamente reglas de asociación entre líneas espectrales que resultaron tener mayor relevancia sobre el conjunto de datos bajo distintas criterios, al aplicarse sobre conjuntos de datos reales obtenidos a partir de observaciones en el espectro visible. Se logró observar conjuntos de especies que están presentes con mayor frecuencia, y las reglas que estos generan; con sus medidas estadísticas correspondientes. Además se logró inferir conclusiones sobre el desempeño de los algoritmos implementados al comparar su eficiencia sobre datos reales con su eficiencia sobre datos simulados.

Queda para desarrollo a futuro una implementación más general del procedimiento para así aplicar los algoritmos a datos obtenidos en otras bandas de frecuencia, como por ejemplo, las observaciones radioastronómicas de ALMA; que por sus caracterísiticas, promete un mayor número de datos sobre los cuales obtener reglas de asociación para líneas espectrales.

\end{intro}
\chapter{Marco Teórico}

% Conceptos involucrados
% - Nuevas tecnologías consideradas
% - Metodologías de desarrollo
% - Algoritmos existentes
% - Arquitecturas estándar
% - Descripción de las soluciones existentes
% Concentra la mayor parte de las citas
% Alquien que conoce del tema podría saltearse este capítulo e igual comprender la memoria

Gran parte de los datos obtenidos desde ALMA son guardados en estructuras de datos llamadas cubos de datos tipo ALMA (o \textit{ALMA Data Cubes}), que contienen información de distintos puntos de observación del cielo a distintas frecuencias. 
Los espectros de frecuencia son una forma de representar la intensidad de la radiación electromagnética, recibida desde un punto del espacio, en un cierto rango de frecuencias. Estos contienen puntos altos de intensidad en ciertas frecuencias en las cuales se sabe que una cierta molécula conocida efectúa una transición cuántica. Por lo tanto, mediante reconocer e identificar estos puntos altos, o \textit{peaks}, se puede saber las transiciones moleculares que ocurrieron en el objeto del que proviene la radiación electromagnética. Como, a su vez, se sabe de antemano a qué frecuencia específicamente se realizan las transiciones cuánticas de moléculas conocidas, puede inferirse cuáles son las moléculas presentes en el objeto de origen.

A su vez, los cubos de datos tipo ALMA, como estructura de datos, contienen valores indexados en tres coordenadas. Dos de las coordenadas son espaciales, y corresponden al equivalente a una imagen normal de dos dimensiones, en el sentido que describen puntos del cielo (o del espacio observable desde la tierra). La tercera coordenada corresponde al rango de frecuencias en el que se está detectando radiación electromagnética. Por lo tanto, si se fijan las dos coordenadas espaciales (se fija un punto en el espacio) y se extraen todos los valores en la tercera coordenada de aquel punto, se obtiene el espectro de frecuencias observado en ese punto del espacio.

Los cubos de datos tipo ALMA, por tanto, contienen información de los espectros de frecuencia observados en todos los puntos de un sector dado del espacio. Todos los espectros presentes en un cubo se encuentran en un mismo rango de frecuencia. Sin embargo, distintos cubos de datos pueden tener observaciones hechas en distintos rangos de frecuencia entre sí.

A partir de ALMA se generan enormes cantidades de datos, los cuales, debido a su gran tamaño,  necesariamente deben procesarse por parte de sistemas automatizados de extracción y análisis con el fin de facilitar a los investigadores el extraer información útil a partir de estos. La mayoría de estas herramientas se encuentran dentro de las áreas de investigación de la minería de datos y Machine Learning; disciplinas de la computación que han tenido un gran auge en el último tiempo.

\section{Reglas de asociación}

Dentro del área del aprendizaje computacional automatizado, o Data Mining, existe una técnica que ha sido ampliamente utilizada e investigada desde su concepción. Se trata del aprendizaje mediante reglas de asociación, o \textit{Association Rule Learning (ARL)}; la cual se creó con el fin de identificar relaciones entre los productos preferidos por los consumidores de distintos sistemas de punto de venta, como supermercados, tiendas de venta al detalle, etc.
La intuición es que, si se posee una base de datos con transacciones, donde cada una de estas posee un cierto conjunto de ítems que un cliente en particular ha comprado, con la ayuda de un algoritmo pueden encontrarse una serie de reglas que indiquen relaciones entre las compras de ciertos ítems en particular. Un ejemplo de regla (bastante intuitiva, por lo demás) sería: “En el 90\% de las transacciones en que se compró pan y mantequilla también se compró leche”. Los algoritmos de ARL permiten obtener relaciones simples, como la del ejemplo, y otras mucho más difíciles de deducir por otros medios.

Formalmente, se considera un conjunto de variables $\mathcal{X}=\{X_{j}\}^p_{j=1}$. Usualmente, estas variables se consideran como binarias: $X_j\in\{0,1\}$. Se define, entonces un conjunto de $N$ \textit{transacciones} $D=\{t_i\}^N_{i=1}$, donde $t_i=\{x_{i,j}\}^p_{j=1}$, y

\[
x_{i,j}=
\left\{
\begin{array}{l l}
0 \quad \text{si es ítem $j$ es parte de la transacción $i$}\\
1 \quad \text{de lo contrario.}
\end{array}
\right.
\]

El objetivo principal del análisis mediante reglas de asociación es obtener aquellos valores de variables conjuntas $(X_1.X_2,\dots,X_p)$ que aparezcan de manera más frecuente en el conjunto de datos.

A partir de éstos se generan reglas de asociación, de la forma:

\[I \Rightarrow X_j\Big|c\]

donde $I\subset\mathcal{X}$, $X_j \notin I$ y $c$ es la \textit{confianza de la regla}, o la razón entre el número de transacciones en $D$ que contienen a $I \cup X_j$ y el número de transacciones que contienen a $I$. A su vez, la razón entre el número de transacciones que contienen a $I \cup X_j$ y el número total de transacciones se conoce como el \textit{soporte} de la regla de asociación.

\chapter{Especificación del Problema}

% Descripción detallada del problema a resolver
% Se discute la relevancia de contar con una solución
% Se especifica en todo detalle los requisitos de la solución a construir
% Características de calidad de la solución deseada
% Es deseable también establecer criterios de aceptación de una solución al problema

\section{Descripción del problema}

Supóngase que se cuenta con conjuntos de espectros, y que cada uno de ellos posee todas sus líneas espectrales correctamente detectadas y, por lo tanto, se conoce su posición en el espectro. En la práctica eso puede ser muy difícil de lograr, sobre todo en circunstancias donde pueden existir en principio una alta cantidad de líneas espectrales y estas pueden interferir unas con otras en la señal final, lo que se conoce como \textit{blending}.

Por lo tanto, para efectos de lo que sigue, basta con asumir que existe la posibilidad que no todas las líneas hayan sido detectadas. Pero es importante que las que sí fueron detectadas, lo hayan sido con una seguridad suficiente y que se sepa de manera adecuada su posición. Actualmente existen herramientas que son capaces de ajustar modelos físicos conocidos con anterioridad a datos espectrales con el fin de identificas las líneas en ellos presentes.

Teniendo estos conjuntos de espectros con sus respectivas líneas detectadas se desea aplicar a conjuntos de líneas espectrales extraídas a partir de datos de observaciones astronómicas, para así obtener información de las relaciones existentes entre ellas bajo distintas medidas de interés y relevancia estadística.

\section{Requisitos de la solución y casos de uso}

A continuación se enuncian los requerimientos del sistema:

\begin{enumerate}
	\item \textbf{Obtener reglas de asociación entre líneas de emisión espectrales [esencial].} \\
	El sistema debe generar reglas de asociación entre líneas de emisión presentes en espectros, independientemente de si estos pertenecen a una misma o a distintas moléculas o átomos, o si no han sido aun identificadas.
	\item \textbf{Permitir al usuario observar las reglas generadas, y ordenarlas según distintas medidas de relevancia estadística [esencial].} \\
	\item \textbf{Permitir al usuario guardar las reglas de asociación generadas [esencial].} \\
	Una vez extraídas las reglas de asociación, el usuario debe poder revisarlas y guardarlas para su revisión posterior.
	\item \textbf{Permitir al usuario aplicar los mismos algoritmos de reglas de asociación a datos de diversas fuentes [esencial].} \\
	Se desea que el sistema de extracción sea lo más general posible, de modo tal de poder aplicarlo a datos de líneas espectrales extraídos de distintos \textit{surveys}, bases de datos, sistemas de modelamiento y detección de líneas, entre otros.
	\item \textbf{El sistema debe ser ejecutable en un ambiente de computación de alto rendimiento [deseable].}
	\item \textbf{El sistema debe ser compatible con plataformas de observatorios virtuales [deseable].} 
	\item \textbf{Implementar una interfaz gráfica de usuario [opcional].}
\end{enumerate}

\subsection{Casos de Uso}

En la Figura \ref{fig:cases} se muestra un diagrama con los casos de uso preliminares del sistema a desarrollar.

\begin{figure}[h!]
\begin{center}
\includegraphics[width=0.9\textwidth]{imagenes/casos_de_uso.png}
\end{center}
\vspace*{-5mm}
\caption{Diagrama de casos de uso del sistema.}
\label{fig:cases}
\end{figure}

\subsubsection{Actores}

Para este sistema existe solo un tipo de actor, dado que todos los usuarios finales tendrán acceso a las mismas funcionalidades. Este usuario será el encargado de seleccionar el conjunto de datos que quiere ingresar al sistema, en forma de transacciones de líneas moleculares. Cada transacción poseerá las líneas identificadas en un espectro en particular. Este usuario ingresará estos datos al sistema y luego seleccionará los parámetros de detección de reglas que desee. Una vez ejecutados los algoritmos correspondientes, el usuario podrá observar las reglas generadas y, si así lo desea, ajustar nuevamente los parámetros para obtener mejores resultados sobre el mismo conjunto de datos.

Desde un punto de vista práctico, el usuario objetivo posee conocimientos técnicos sobre espectroscopía, sabe hacer uso de un terminal o línea de comandos, y puede manejar tablas en formato de valores separados por comas (CSV).


\subsubsection{Descripción de casos de uso}

En la siguiente tabla se muestra una descripción detallada de los casos de uso y se indica, de ser así, a qué requerimiento está asociado.

\begin{tabular}{|l|p{4cm}|p{7cm}|l|l|}
	\hline
	ID & Caso de uso & Descripción & Tipo & Ref. \\ \hline
	1 & Obtener reglas de asociación generales & El usuario obtiene reglas de asociación extraídas a partir de un conjunto de transacciones de líneas espectrales y las filtra u ordena mediante soporte, confianza o \textit{lift} & Esencial & 1,2,3,4 \\ \hline
	2 & Obtener reglas de asociación de un cierto ítem & El usuario obtiene reglas de asociación extraídas a partir de un conjunto de transacciones de líneas espectrales, selecciona solo aquellas que posean un cierto ítem en su antecedente y/o consecuente, y las ordena mediante soporte, confianza o \textit{lift}. & Esencial & 1,2,3,4 \\ \hline
\end{tabular}

\chapter{Descripción de la Solución}

% Involucra todos o algunos de los siguientes:
% - Arquitectura del software
% - Arquitectura del hardware
% - Diseño de la base de datos
% - Diseño de clases
% - Diseño de estructuras de datos
% - Diseño de algoritmos
% - Diseño de la interfaz de usuario
% Justificación y descripción de cómo la solución resuelve el problema planteado

A continuación se describe la solución implementada para el presente proyecto. Se detalla aquí la estructura, diseño y funcionamiento del sistema y la aplicación realizados con el fin de cumplir con los requerimientos descritos anteriormente.

\section{Arquitectura de software}

Dado que, para fines del proyecto, se requería de una herramienta con la cual se puediese llevar a cabo una serie de pruebas en distintos contextos, se optó por dividir el sistema en dos paquetes distintos; cada uno con una función específica, e interfaces bien definidas, con el fin de facilitar su posterior extensión y reutilización.

\missingfigure{Diagrama de los paquetes del sistema}

A continuación se detallan los paquetes del sistema, sus módulos, interfaces y funciones:

\subsection{Paquete de Association Rule Learning (ARL)}

El paquete de Association Rule Learning (ARL) es el encargado de realizar el aprendizaje mediante reglas de asociación en sí; vale decir, de recibir un conjunto de datos con transacciones y de retornar reglas de asociación generadas a partir de aquel conjunto.

En las siguientes secciones se espacifican los formatos de entrada y salida de este paquete junto con una descripción de los módulos que lo componen.

\subsubsection{Módulo de interfaz de usuario/controlador}

El módulo de interfaz de usuario y controlador es el encargado de recibir directamente del usuario los parámetros de entrada correspondientes. Este módulo contiene métodos, clases y funciones que reciben los parámetros del usuario, abren y leen los archivos de entrada adecuados, los procesan de acuerdo al formato especificado, y hacen entrega de los datos al módulo principal de ARL.

Este módulo, es el encargado, además de recibir las reglas de asociación, entregarlas al módulo de formato para luego retornarlas al usuario en un archivo correspondiente.

\subsubsection{Módulo de formato}

Es el módulo encargado de analizar los archivos de entrada leídos por el módulo de interfaz de usuario, extraer la información pertinente de ellos según el formato especificado, y retornar los datos en una estructura adecuada para luego ser procesados por el módulo principal de ARL. A su vez, este módulo realiza, además la labor inversa; vale decir, recibe las reglas de asociación en una estructura de datos estándar para luego entregarlas al módulo de interfaz en el formato requerido por el usuario.

Hasta el momento los formatos soportados son CSV para archivos de entrada, y CSV o tabla en formato \LaTeX\ para archivos de salida.

\subsubsection{Módulo principal de ARL}

El módulo principal de ARL es el encargado de llevar a cabo el algoritmo de aprendizaje mediante reglas de asociación en sí. En su parte lógica, consta de dos sub-módulos principales. El primero es es sub-módulo encargado de extraer los conjuntos de ítemes frecuentes; vale decir, aquellos que cumplen con el requerimiento de soporte mínimo. Y el segundo es el sub-módulo de generación de reglas, que es el encargado de recibir los conjuntos de ítemes frecuentes y generar, a partir de ellos, las reglas de asociación que cumplen con el requerimiento de confianza mínima indicado.

\subsubsection{Módulo de testeo de ARL}

Se encuentra dentro de este paquete, además, un módulo de testeo de los algoritmos de ARL sobre datos de prueba de pequeña envergadura; con el fin de realizar chequeos periódicos del funcionamiento correcto de estos algoritmos en la medida que se realizan cambios, mejoras o refactorizaciones sobre su código fuente.

\subsubsection{Módulo de herramientas}

Finalmente, se encuentra el módulo de herramientas generales, que consta de una serie de funciones de uso frecuente por parte de otros módulos del paquete; tales como operaciones sobre listas anidadas, búsqueda de llaves sobre diccionarios específicos, entre otros.

\subsection{Paquete de procesamiento de datos}

Debido a que, en la mayoría de las ocasiones los datos sobre los cuales se desea aplicar los algoritmos de reglas de asociación no se encuentran desde un comienzo en los formatos o estructuras necesarias, se procedió a implementar un paquete de pre-procesamiento. Este contiene una serie de scripts y métodos cuya función principal es extraer los datos desde sus fuentes originales, opcionalmente inferir aquella información que sea relevante, y guardarla en archivos cuyo formato sea comprensible para el paquete de aprendizaje de reglas de asociación.

En su implementación actual, este paquete se encuentra enfocado, en su mayor parte, para trabajar sobre datos extraídos a partir del Sloan Digital Sky Survey (SDSS).

A continuación se enumeran algunos de sus componentes más importantes.

\subsubsection{Queries SQL}

Una colección de queries relevantes para ejecutar en las bases de datos de SDSS y extraer los datos sobre los cuales obtener las reglas de asociación.

\subsubsection{Módulo de procesamiento de tablas}

Contiene una serie de scripts cuyo fin es recibir un archivo de tabla de base de datos en formato CSV y procesar los datos que contiene; por ejemplo, eliminando ciertas filas, añadiendo columnas calculadas a partir de las ya existentes, entre otros. Los resultados son guardados en un nuevo archivo de tabla en formato CSV.

\subsubsection{Módulo de extracción de transacciones}

Este módulo contiene scripts cuya función es recibir un archivo de tabla de base de datos en formato CSV, y a partir de él generar un archivo CSV que contenga una transacción por cada fila; cada una de estas con una lista de ítemes en formato adecuado para ser recibido por el paquete de ARL.

\section{Diseño de clases}

A continuación se detallan las clases de objetos más importantes del sistema.

\missingfigure{Diagrama UML de las clases más importantes del sistema}

\subsection{Clase \textit{ItemSet}}

La clase \textit{ItemSet} es la encargada de mantener información sobre un conjunto de ítemes y abstraer su estructura de datos subyacente. Cada instancia de esta clase corresponde a un conjunto de ítemes distinto, y contiene campos que guardan la información más reciente sobre su soporte (calculado sobre un cierto conjunto de transacciones) y punteros a meta-datos con información adicional sobre los ítemes en sí. Su interfaz asegura que se pueda realizar de forma adecuada, visto desde un punto de vista matemáticamente abstracto, las operaciones más comúnes de conjuntos de elementos; como comprobar pertenencia, sumar de conjuntos, diferencia entre conjuntos, entre otros.

\subsection{Clase \textit{AssociationRule}}

La clase \textit{AssociationRule} es la que define la estructura y comportamiento de las reglas de asociación. Cada instancia de esta clase corresponde a una regla de asociación en particular, extraída a partir de un cierto conjunto de datos. Cada regla de asociación consta de dos objetos de la clase \textit{ItemSet}; uno para el antecedente y otro para el consecuente de la regla. Además contiene un campo que codifica su soporte, junto con métodos para calcular sus medidas de relevancia, tales como su confianza y lift.

\subsection{Clase \textit{FrequentItemSetMiner}}

La clase \textit{FrequentItemSetMiner} es la encargada de abstraer y guardar información sobre el proceso de extraer a partir de las transacciones aquellos conjuntos de ítemes que cumplan con un requisito de soporte mínimo dado. Cada instancia de esta clase corresponde a un proceso de estracción distinto, conteniendo campos y estructuras de datos para los algoritmos involucrados, su estado actual y su resultado.

En su implementación actual, esta clase es heredada por dos sub-clases. Una correspondiente al algoritmo \textit{Apriori}, y otra al algoritmo \textit{FP-Growth}. Cada una contiene su propia implementación de los métodos principales, definidos en su clase padre, junto con sus propias funciones auxiliares y estructuras de datos correspondientes.

\subsection{Clase \textit{RuleMiner}}

La clase \textit{RuleMiner} es la que abstrae el proceso de extraer reglas de asociación a partir de conjuntos frecuentes de ítemes. Cada instancia de esta clase corresponde a un proceso de extracción distinto; básicamente el mismo en todo los casos salvo en ciertos detalles, como algunas funciones auxiliares y referencias a estructuras de datos, dependiendo de si los conjuntos fueron extraídos mediante \textit{Apriori} o \textit{FP-Growth}.

\section{Detalles de implementación}

La implementación del sistema se llevó a cabo en el lenguaje de programación Python. Se realizó una implementación propia de los algoritmos antes descritos, con algunas adaptaciones para su funcionamiento correcto en el contexto de este proyecto; y se hizo uso de paquetes externos con el fin de hacer más simple el manejo de archivos CSV y la implementación de la interfaz por línea de comando.

\subsection{Extracción de conjuntos de ítemes frecuentes}

Para la extracción de conjuntos de ítemes frecuentes se procedió a realizar la implementación de los algoritmos \textit{Apriori} y \textit{FP-Growth}. Ambos algoritmos reciben las transacciones en una misma estructura de datos y retornan los conjuntos frecuentes también en una misma estructura en ambos casos. Pero cada una de estas clases posee sus propios métodos, definidos por los algoritmos en general.

En general, para ambos algoritmos la estructura de datos más utilizada para la implementación subyacente en los objetos correspondientes a conjuntos frecuentes, candidatos, antecedentes y consecuentes por igual, fue la de \textit{frozensets}. Esta clase de objetos, además de permitir las operaciones matemáticas de conjuntos clásicas, tales como sumas y diferencias de conjuntos, permite que los objetos sean hasheables; y, por lo tanto, utilizar los conjuntos como llaves de diccionario en forma de tablas de hash, y de esta forma, por ejemplo, indexar por conjunto distintas estructuras de datos auxiliares.

\subsection{Extracción de reglas de asociación}

La extracción de reglas de asociación a partir de conjuntos frecuentes se llevó a cabo mediante una implementación del algoritmo \textit{Apriori} de generación de reglas.

\subsubsection{Entrada y salida}

El paquete de \textit{Association Rule Learning (ARL)} recibe como entrada un archivo de tabla en formato de valores separados por coma o \textit{comma separated values (CSV)}. Este archivo debe tener el siguiente formato en cada una de sus filas

\begin{lstlisting}[basicstyle=\ttfamily]
<TID>,"<ItemList>"
\end{lstlisting}

donde \textit{<TID>} es el identificador de la presente transacción, e \textit{<ItemList>} es una lista de identificadores únicos de los ítemes presentes en la transacción separados por comas. Tal como se indica, esta lista debe ir rodeada por comillas dobles en el archivo de entrada. A continuación se muestra un ejemplo de archivo de entrada válido.

\begin{lstlisting}[basicstyle=\ttfamily]
000001,"15,2,44"
000002,"5,4,23,67,43,234"
000003,"66,3,53,23"
\end{lstlisting}

Adicionalmente, se puede especificar para cada transacción un tipo o clase a la que pertenece, o de la cual se origina, con el fin de realizar estadísticas pertinentes con las reglas generadas. De ser así, el archivo de entrada debe tener el siguiente formato en cada una de sus filas,

\begin{lstlisting}[basicstyle=\ttfamily]
<TID>,<Class>,"<ItemList>"
\end{lstlisting}

donde, en esta ocasión, se añade en la segunda posición el campo \textit{<Class>}, que consiste en una secuencia de caracteres válidos que identifique de manera unívoca la clase a la cual la transacción pertenece. A continuación un ejemplo de entrada válida en este formato.

\begin{lstlisting}[basicstyle=\ttfamily]
000001,MORNING,"15,2,44"
000002,MORNING,"5,4,23,67,43,234"
000003,NIGHT,"66,3,53,23"
\end{lstlisting}

Esta lista es leída y procesada dentro del paquete de ARL y luego entregada en una estructura de datos correspondiente al algoritmo indicado, que obtendrá las reglas de asociación presentes en el conjunto de transacciones. Estas reglas, por defecto, serán retornadas en un archivo de texto en formato CSV con la siguiente estructura en cada una de sus líneas.

\begin{lstlisting}[basicstyle=\ttfamily]
<N>,"<Antecedent>","<Consequent>",<Support>,<Confidence>,<Lift>
\end{lstlisting}

Donde \textit{N} es un número identificador de la regla de asociación, \textit{<Antecedent>} es una lista de ítemes separados por coma correspondientes al antecedente de la regla, \textit{<Consequent>} es una lista de ítemes separados por coma correspondientes al consecuente de la regla, \textit{<Support>} es un valor de punto flotante entre 0 y 1 correspondiente al soporte de la regla, \textit{<Confidence>} es un valor de punto flotante entre 0 y 1 correspondiente a la confianza de la regla, y \textit{<Lift>} es un valor de punto flotante entre 0 y 1 correspondiente al lift de la regla. A continuación un ejemplo de este formato de archivo de salida.

\begin{lstlisting}[basicstyle=\ttfamily]
1,"15,33","2,89,91",0.21,0.85,2.31
2,"12,33,44","5,23,31",0.23,0.81,3.3
\end{lstlisting}

Si, además, en los datos de entrada se especificó una clase para cada transacción, entonces el archivo de salida tendrá el siguiente formato

\begin{lstlisting}[basicstyle=\ttfamily]
<N>,"<Antecedent>","<Consequent>",<Support>,<Confidence>,<Lift>,"<ClassCount>"
\end{lstlisting}

en donde \textit{<ClassCount>} es una lista de valores separados por comas con el siguiente formato

\begin{lstlisting}[basicstyle=\ttfamily]
<Class01>:<Count01>,<Class02>:<Count02>,...
\end{lstlisting}

donde \textit{<Class01>} es el identificador de la primera clase, \textit{<Count01>} es un número entero que indica cuántas de las transacciones que satisfacen la regla actual pertenecen a esta primera clase, y así sucesivamente con todas las clases posibles. A continuación un ejemplo de archivo de salida con el formato recién descrito.

\begin{lstlisting}[basicstyle=\ttfamily]
1,"15,33","2,89,91",0.21,0.85,2.31,"MORNING:210,NIGHT:15"
2,"12,33,44","5,23,31",0.23,0.81,3.3,"MORNING:20,NIGHT:91"
\end{lstlisting}


\section{Interfaz de usuario}




\chapter{Validación de la Solución}

% Demostración de cómo la solución resuelve el problema
% Dependiendo de la naturaleza del problema / solución:
% - Uso de la aplicación desarrollada en un contexto real reportando los resultados
% - Simulación de uso con un caso representativo
% - Encuesta a usuarios finales
% Dependiendo de la longitud, la validación puede ser una sección al final del capítulo de solución o un capítulo independiente

\begin{conclusion}

% Breve resumen del trabajo realizado
% Recuento de objetivos alcanzados y no alcanzados
% Análisis crítico de por qué los resultados fueron los reportados
% Reflexión acerca de la relevancia / impacto del trabajo realizado
% Lecciones aprendidas
% Posibles trabajos futuros que podrían hacerse a partir de la memoria para mejorar aún más la solución


\end{conclusion}

\nocite{*}
\bibliographystyle{plain}
\bibliography{report}
\end{document}
